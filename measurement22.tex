% This is samplepaper.tex, a sample chapter demonstrating the
% LLNCS macro package for Springer Computer Science proceedings;
% Version 2.20 of 2017/10/04
%
\documentclass[runningheads]{llncs}
%
\usepackage{graphicx}
% Used for displaying a sample figure. If possible, figure files should
% be included in EPS format.
%\usepackage{tikz}
%\usetikzlibrary{arrows}
\usepackage{verbatim}
%\usepackage{amsmath}
%\usepackage{amssymb}
%\usepackage{graphicx}
%\usepackage[all]{xy}
\usepackage{array}
\usepackage{enumitem}
%\usepackage{cite}
\usepackage{natbib}
% If you use the hyperref package, please uncomment the following line
% to display URLs in blue roman font according to Springer's eBook style:
% \renewcommand\UrlFont{\color{blue}\rmfamily}
\usepackage[breaklinks=true]{hyperref}
\usepackage{breakcites}
\renewcommand\UrlFont{\color{blue}\rmfamily}

\begin{document}
%
\title{Runtime Measurement White Paper}
%
%\titlerunning{Abbreviated paper title}
% If the paper title is too long for the running head, you can set
% an abbreviated paper title here
%
\author{Yan Shoshitaishvili\inst{1} \and Perry Alexander\inst{2}}
%
\authorrunning{Y. Shoshitaishvili and P. Alexander}
% First names are abbreviated in the running head.
% If there are more than two authors, 'et al.' is used.
%
\institute{Arizona State University \\
  Tempe, AZ \\
  \email{yans@asu.edu}
  \and
  Institute for Information Sciences \\ The
  University of Kansas \\ Lawrence, KS 66045 \\
  \email{palexand@ku.edu}}
%
\maketitle              % typeset the header of the contribution
%
\begin{abstract}
The abstract should briefly summarize the contents of the paper in
15--250 words.

\keywords{First keyword  \and Second keyword \and Another keyword.}
\end{abstract}
%
%
%
\section{First Section}

Use fuzzing, compilation and static analysis techniques to guide generation of run-time measurement and appraisal routines.

The University of Kansas has developed mechanisms for executing and appraising measurement and meta-measurement in support of trust.  Arizona State University has developed mechanisms for fuzzing and static analysis of the Linux heap and user space programs.  We propose using the results of fuzzing and static analysis to identify memory regions whose measurements will be particularly informative.

Fuzzing to generate and identify "weird" memory states.  Develop measurements to detect these memory states.  Because we understand how fuzzing is performed we understand what measurements tell us about observed memory.

Program decompilation to identify critical system variables for runtime measurement.  Decompilation produces source with variables present.  Perform source-level analysis to determine what variables are valuable for runtime measurement.  Synthesize measurers and appraisal routines for variable memory locations.  (Use type information to bound memory values?)

Use static analysis and compilation techniques to automatically generate measurement and appraisal routines from OS and user-space software.
Identify cases where dynamic allocators are not robust to other bugs. Lookat the heap using bounded model checking.  Look for pacial memory errors on the heap that could identify measurement targets.

Compile in runtime instrumentation to enhance measurement.  "Handles" for memory locations, markers for locations and events, runtime monitors like trip-wires.  Open research problem.

% 
% 
% ---- Bibliography ----
%
% BibTeX users should specify bibliography style 'splncs04'.
% References will then be sorted and formatted in the correct style.
%
%\bibliographystyle{splncs04}
\bibliographystyle{splncsnat}
%\bibliography{sldg}
\bibliography{bib/sldg}
%
\end{document}
